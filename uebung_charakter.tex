% Created 2014-04-11 Fr 13:32
\documentclass[11pt]{article}
\usepackage[utf8]{inputenc}
\usepackage[T1]{fontenc}
\usepackage{graphicx}
\usepackage{longtable}
\usepackage{float}
\usepackage{wrapfig}
\usepackage{soul}
\usepackage{amssymb}
\usepackage{hyperref}


\title{uebung$_{\mathrm{charakter}}$}
\author{Sebastian Domaschke}
\date{11 April 2014}

\begin{document}

\maketitle

\setcounter{tocdepth}{3}
\tableofcontents
\vspace*{1cm}
\section{Character}
\label{sec-1}

\subsection{Definition}
\label{sec-1.1}

\begin{description}
\item Character(<int>):: ch
\item Character(len=<int>):: ch
\item Character:: ch !Ch hat die Länge 1
     z.B.

\begin{description}
\item character(4)::ch
\item Integer, Parameter::n=16
\item character(n)::ch
\end{description}

\end{description}
\subsection{Bedeutung:"*"}
\label{sec-1.2}

\begin{itemize}
\item falls character paramter ist
\item falls character eingabeparameter ist
     z.B.

\begin{itemize}
\item character(*), parameter::ch='Haus',c='aus'
\item Achtung: c hat auch Länge 4 -> c='aus_'
\item Function(c)
       character(x)::c
\end{itemize}

\end{itemize}
\subsection{Operatoren}
\label{sec-1.3}

\begin{itemize}
\item \emph{/ Konkatenation
\item `Eins'/\}'Zwei'->'EinsZwei'
\item Vergleichopearatoren <,>,<=,>=,==,/=
     ->basierend auf ASCII-Code
     ->endständige Leerzeichen, unberücksichtg
     Bsp `Eins'=='Eins__'=.true.
\end{itemize}
\subsection{Zugriff:}
\label{sec-1.4}

\begin{itemize}
\item character der Länge n verhält sich wie ein Feld der Länge n von character(1)-Variablen
     Bsp: Character(10)::ch='Hallo Welt'
     ch(2:4)  !all
     ch(:5)  !hallo
     ch(7:)  !welt
     ch(7:)='XYZ'  !ch=hallo$_{\mathrm{XYZ}}$\_{}
\end{itemize}
\subsection{Funktionen}
\label{sec-1.5}

   character(10)::ch='Hallo du'
   character(5)::dh='Hallo'
\subsection{Zuweisung:}
\label{sec-1.6}

   ch=dh -> ch = `Hallo_\ldots{}_'
   dh='Funktion'->dh='Funkt'
\begin{itemize}
\item LEN(c)-> Len(ch)=10
\item LEN$_{\mathrm{TRIM}}$(ch)->LEN$_{\mathrm{TRIM}}$(ch)=8
\item TRIM(c): TRIM(ch)='Hallo du'
\item ADJUSTL(c): ADJUSTL(`__abc')='abc__'
\item ADJUSTR(c): ADJUSTR(ch)='__Hallo$_{\mathrm{du}}$'
\item REPEAT(c,i):Repeat(`abc',2)='abcabc'
\item weitere:INDEX,SCAN,VERIFY
\end{itemize}
\subsection{Konvertierung von Character zu Integer}
\label{sec-1.7}

\subsubsection{I}
\label{sec-1.7.1}

     -WRITE(ch,*);(schreibt Inhalt von i in ch)
     -READ(ch,*); (liest Wert aus ch)
\subsubsection{II über Funktionen}
\label{sec-1.7.2}

    CHAR(i),ACHAR(i) geben zu i zugeordnetes Zeichen zurück nach ASCII
    ICHAR(ch),IACHAR(ch) geben zu ch zugeordenten wert zurück nach ASCII
    -> über DO-Schleife, vergl. Klausuraufgabe    

\end{document}